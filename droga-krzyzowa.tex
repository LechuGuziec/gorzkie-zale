%\documentclass[wide,a4paper,12pt,titlepage]{mwart}
\documentclass[wide,landscape,a4paper,12pt]{mwart}

\usepackage[utf8]{inputenc}
\usepackage[T1]{fontenc}
\usepackage{polski}
\usepackage{lmodern}
\usepackage{graphicx,pdflscape}
\usepackage{listings}
\usepackage{setspace}
%\linespread{1.1} %interlinia
\setlength{\parskip}{1ex plus 0.5ex minus 0.2ex} % Odstępy akapitów
\setlength\parindent{0pt}
\usepackage{geometry}
\geometry{verbose,a4paper,tmargin=1.5cm,bmargin=1.5cm,lmargin=1cm,rmargin=1cm} % Marginesy strony
\setcounter{secnumdepth}{0} % Brak numeracji nagłówków
%\usepackage[osf]{libertine} % Font
\usepackage{multicol}
\setlength\columnsep{2cm}
%\usepackage{showframe}
%\renewcommand*\ShowFrameColor{\color{red}}

%Strona tytułowa


\begin{document}
\setlength{\columnseprule}{0.1pt}

\pagestyle{empty}
\begin{multicols}{2}
	
\section{Droga Krzyżowa}

\subsection{Stacja I. Jezus na śmierć skazany}
Zadrżyj ma duszo, oto Stwórca świata został skazany na śmierć przez Piłata! Za twoje grzechy życie swe Syn Boży na krzyżu łoży.

\subsection{Stacja II. Jezus bierze krzyż na swe ramiona}
Jezus ze drżeniem bierze krzyż w ramiona, usty całuje, chociaż na nim skona;
lecz swoją śmiercią imię Ojca wsławi, a ludzi zbawi.

\subsection{Stacja III. Jezus po raz pierwszy upada pod krzyżem }
Jezus pod krzyżem z trudu już omdlewa, na twarz upada pod ciężarem drzewa.
Znęca się nad nim i przekleństwa miota żołdaków rota.

\subsection{Stacja IV. Jezus spotyka swą Matkę}
Boleść przebiła tkliwe Matki serce, kiedy ujrzała Syna w poniewierce.
Milczy w boleści, tylko łzy mówiły: Synu mój miły!

\subsection{Stacja V. Jezus korzysta z pomocy Szymona Cyrenejczyka}
Jezus z wdzięcznością patrzy na Szymona, że z Nim krzyż ciężki wziął na swe ramiona.
Błogosławiony, że się z Panem trudzi, by zbawić ludzi.

\subsection{Stacja VI. Jezusowi pomaga Weronika}
Litością tknięta córka Izraela chustą ociera lice Zbawiciela.
Jezus ją w zamian wizerunkiem darzy swej świętej twarzy.

\subsection{Stacja VII. Jezus upada po raz wtóry}
O Jezu mity, znów upadasz srodze, leżysz pod krzyżem na golgockiej drodze.
Nie brzemię drzewa Ciebie tak przygniata, lecz grzechy świata.


\subsection{Stacja VIII Jezus poucza płaczące niewiasty}
Z płaczem niewiasty załamuję ręce, tkliwie współczuję Jezusowej męce.
On je pociesza, iż to cierpienie da im zbawienie.

\subsection{Stacja IX. Jezus upada pod krzyżem po raz trzeci}
Przypatrz się duszo, jak się Jezus słania i po raz trzeci pada z wyczerpania.
Leży jak robak w prochu przydeptany, sponiewierany.


\subsection{Stacja X. Jezus z szat obnażony}
Stwórca, co niebo złotą zorzą stroi, wstydem okryty wobec ludzi stoi!
Pan wszechstworzenia, co odziewa kwiaty, stoi bez szaty.

\subsection{Stacja XI. Chrystus do krzyża przybity}
Ciało Jezusa ból przenika srogi, gdy gwoździe biją w obie ręce, nogi.
Jak struna lutni drży na całym ciele, cierpiąc zbyt wiele.


\subsection{Stacja XII. Chrystus na krzyżu umiera}
Jezus na krzyżu po skończonej męce Ducha oddaje w Ojca swego ręce.
Drży ziemia z trwogi, słońce blask zawiera, gdy Bóg umiera.

\subsection{Stacja XIII. Jezus z krzyża zdjęty}
Siedzi pod krzyżem Matka nieszczęśliwa, z ran swego Syna łzami krew obmywa.
0 święta Matko, swej boleśni łzami módl się za nami!

\subsection{Stacja XIV. Jezus złożony do grobu}
W grobie złożono Jezusowe ciało, które za ludzi tyle wycierpiało.
Lecz dnia trzeciego, gdy przyjdzie świtanie, znów żywe wstanie!

\end{multicols}

\end{document}
