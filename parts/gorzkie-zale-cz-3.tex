\section{CZĘŚĆ TRZECIA}


\subsection{Pobudka}

Gorzkie żale przybywajcie,\\ Serca nasze przenikajcie, Serca nasze przenikajcie

Rozpłyńcie się me źrenice,\\ 
Toczcie smutnych łez krynice, Toczcie smutnych łez krynice.

Słońce, gwiazdy omdlewają,\\ 
Żałobą się pokrywają, Żałobą się pokrywają, 

Płaczą rzewnie Aniołowie, \\ 
A któż żałość ich wypowie, A któż żałość ich wypowie?

Opoki się twarde krają,\\ 
Z grobów umarli powstają, Z grobów umarli powstają.

Cóż jest, pytam, co się dzieje?\\
Wszystko stworzenie truchleje! Wszystko stworzenie truchleje!

Na ból Męki Chrystusowej,\\ 
Żal przejmuje bez wymowy, Żal przejmuje bez wymowy.

Uderz, Jezu, bez odwłoki,\\
W twarde serc naszych opoki!  W twarde serc naszych opoki!

Jezu mój, we krwi ran Swoich,\\
Obmyj duszę z grzechów moich! Obmyj duszę z grzechów moich! 

Upał serca swego chłodzę,\\
Gdy w przepaść Męki Twej wchodzę.  Gdy w przepaść Męki Twej wchodzę.


\subsection{Intencja}

W ostatniej części będziemy rozważali, co Pan Jezus cierpiał od chwili
ukoronowania aż do ciężkiego skonania na krzyżu. Te bluźnierstwa,
zelżywości i zniewagi, jakie Mu wyrządzono, ofiarujemy za grzeszników
zatwardziałych, aby Zbawiciel pobudził ich serca zbłąkane do pokuty i
prawdziwej poprawy życia, oraz za dusze w czyśćcu cierpiące, aby im
litościwy Jezus Krwią swoją świętą ogień zgasił; prośmy nadto, by i nam
wyjednał na godzinę śmierci skruchę za grzechy i szczęśliwe w łasce
Bożej wytrwanie.


\subsection{Hymn}

Duszo oziębła, czemu nie gorejesz?\\
Serce me, czemu całe nie truchlejesz?\\
Toczy twój Jezus z ognistej miłości\\
Krew w obfitości.

Ogień miłości, gdy Go tak rozpala,\\
Sromotne drzewo na ramiona zwala;\\
Zemdlony Jezus pod krzyżem uklęka,\\
Jęczy i stęka.\\
Okrutnym katom posłuszny się staje,\\
Ręce i nogi przebić sobie daje,\\
Wisi na krzyżu, ból ponosi srogi\\
Nasz Zbawca drogi.

O słodkie drzewo, spuśćże nam już ciało,\\
Aby na tobie dłużej nie wisiało!\\
My je uczciwie w grobie położymy,\\
Płacz uczynimy.

Oby się serce we łzy rozpływało,\\
Że Cię, mój Jezu, sprośnie obrażało!\\
Żal mi, ach żal mi ciężkich moich złości\\
Dla Twej miłości!

Niech Ci, mój Jezu, cześć będzie w wieczności\\
Za Twe obelgi, męki, zelżywości,\\
Któreś ochotnie, Syn Boga jedyny,\\
Cierpiał bez winy!


\subsection{Lament duszy nad cierpiącym Jezusem}

Jezu, od pospólstwa niezbożnie, Jako złoczyńca z łotry porównany, Jezu
mój kochany!\\
Jezu, od Piłata niesłusznie, Na śmierć krzyżową za ludzi skazany, Jezu
mój kochany!\\
Jezu, srogim krzyża ciężarem, Na kalwaryjskiej drodze zmordowany, Jezu
mój kochany!\\
Jezu, do sromotnego drzewa, Przytępionymi gwoźdźmi przykowany, Jezu mój
kochany!\\
Jezu, jawnie pośród dwu łotrów, Na drzewie hańby ukrzyżowany, Jezu mój
kochany!\\
Jezu, od stojących wokoło, I od przechodzących szyderczo wyśmiany, Jezu
mój kochany!\\
Jezu, bluźnierstwami od złego, Współwiszącego łotra wyszydzany, Jezu mój
kochany!\\
Jezu, gorzką żółcią i octem, W wielkim pragnieniu swoim napawany, Jezu
mój kochany!\\
Jezu, w swej miłości niezmiernej, Jeszcze po śmierci włócznią przeorany,
Jezu mój kochany!\\
Jezu, od Józefa uczciwie, I od Nikodema w grobie pochowany, Jezu mój
kochany!

Bądź pozdrowiony, bądź pochwalony, dla nas zelżony i pohańbiony!\\
Bądź uwielbiony, bądź wysławiony, Boże nieskończony!


\subsection{Rozmowa duszy z Matką Bolesną}

Ach, ja Matka boleściwa, Pod krzyżem stoję smutliwa, Serce żałość
przejmuje.\\
O Matko, niechaj prawdziwie, Patrząc na krzyż żałośliwie, Płaczę z Tobą
rzewliwie!\\
Jużci, już moje Kochanie, Gotuje się na konanie! Toć i ja z Nim
umieram!\\
Pragnę, Matko zostać z Tobą, Dzielić się Twoją żałobą, Śmierci Syna
Twojego.\\
Zamknął słodką Jezus mowę, Już ku ziemi skłania głowę, Żegna już Matkę
swoją!\\
O Maryjo, Ciebie proszę, Niech Jezusa rany noszę, I serdecznie rozważam.

\emph{Któryś za nas cierpiał rany,\\ Jezu Chryste zmiłuj się nad nami!} x3



